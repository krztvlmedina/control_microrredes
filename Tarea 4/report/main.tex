% Template:     Informe LaTeX
% Documento:    Archivo principal
% Versión:      7.0.0 (23/06/2020)
% Codificación: UTF-8
%
% Autor: Pablo Pizarro R.
%        Facultad de Ciencias Físicas y Matemáticas
%        Universidad de Chile
%        pablo@ppizarror.com
%
% Manual template: [https://latex.ppizarror.com/informe]
% Licencia MIT:    [https://opensource.org/licenses/MIT]

% CREACIÓN DEL DOCUMENTO
\documentclass[letterpaper,oneside]{article}

% INFORMACIÓN DEL DOCUMENTO
\def\titulodelinforme {Tarea 4}
\def\temaatratar {Implementación de DMPC para una micro-red DC}

\def\autordeldocumento {Cristóbal.}
\def\nombredelcurso {Control Avanzado de Micro-redes}
\def\codigodelcurso {EL7058-1}

\def\nombreuniversidad {Universidad de Chile}
\def\nombrefacultad {Facultad de Ciencias Físicas y Matemáticas}
\def\departamentouniversidad {Departamento de Ingeniería Eléctrica}
\def\imagendepartamento {logos/die}
\def\imagendepartamentoescala {0.2}
\def\localizacionuniversidad {Santiago, Chile}

% INTEGRANTES, PROFESORES Y FECHAS
\def\tablaintegrantes {
\begin{tabular}{ll}
	Integrantes:
	& \begin{tabular}[t]{l}
		Cristóbal Medina M. \\
	\end{tabular} \\
	Profesores:
	& \begin{tabular}[t]{l}
		Alex Navas\\
	\end{tabular} \\
	Auxiliares:
	& \begin{tabular}[t]{l}
		Benjamín Moreno V. \\
		Matías Alegría S.\\
		
	\end{tabular} \\

	\multicolumn{2}{l}{Fecha de entrega: 29 de Noviembre 2024} \\
	\multicolumn{2}{l}{\localizacionuniversidad}
\end{tabular}}{
}

% IMPORTACIÓN DEL TEMPLATE
\input{template}

% INICIO DE PÁGINAS
\begin{document}
	
% PORTADA
\templatePortrait

% CONFIGURACIÓN DE PÁGINA Y ENCABEZADOS
\templatePagecfg

% RESUMEN O ABSTRACT


% TABLA DE CONTENIDOS - ÍNDICE
\templateIndex

% CONFIGURACIONES FINALES
\templateFinalcfg

% ======================= INICIO DEL DOCUMENTO =======================

% Template:     Informe LaTeX
% Documento:    Archivo de ejemplo
% Versión:      7.0.0 (23/06/2020)
% Codificación: UTF-8
%
% Autor: Pablo Pizarro R.
%        Facultad de Ciencias Físicas y Matemáticas
%        Universidad de Chile
%        pablo@ppizarror.com
%
% Manual template: [https://latex.ppizarror.com/informe]
% Licencia MIT:    [https://opensource.org/licenses/MIT]

% ------------------------------------------------------------------------------
% NUEVA SECCIÓN
% ------------------------------------------------------------------------------
% Las secciones se inician con \section, si se quiere una sección sin número se
% pueden usar las funciones \sectionanum (sección sin número) o la función
% \sectionanumnoi para crear el mismo título sin numerar y sin aparecer en el índice

\newcommand{\explorelite}{\textit{explore\_lite}}
\newcommand{\movebase}{\textit{move\_base}}

\section{Introducción}
 % Ejemplo, se puede borrar

% FIN DEL DOCUMENTO
\end{document}
